\documentclass[11pt]{article}
 \pdfminorversion=4
\usepackage[utf8]{inputenc}
\usepackage{amsfonts,epsfig}
\usepackage[hyphens]{url}
%\usepackage{hyperref}
\usepackage{breakurl}
\usepackage{comment}

\usepackage{natbib}

\linespread{1.25}

\usepackage{relsize}
\usepackage{fancyvrb}
\usepackage{amsmath, amssymb}
%%% Document layout, margins
\usepackage{geometry}
\geometry{letterpaper, textwidth=6.5in, textheight=9in, marginparsep=1em}
%%% Section headings
\usepackage{sectsty}
\usepackage{caption}
\usepackage{subcaption}
\usepackage[normalem]{ulem}
\sectionfont{\sffamily\bfseries\upshape\large}
\subsectionfont{\sffamily\bfseries\upshape\normalsize}
\subsubsectionfont{\sffamily\mdseries\upshape\normalsize}
\makeatletter
\renewcommand\@seccntformat[1]{\csname the#1\endcsname.\quad}

\makeatletter
\def\@maketitle{%
  \begin{center}%
  \let \footnote \thanks
    {\large \@title \par}%
    {\normalsize
      \begin{tabular}[t]{c}%
        \@author
      \end{tabular}\par}%
    {\small \@date}%
  \end{center}%
}
\makeatother


\title{\bf Bayesian leave-one-out cross-validation for non-factorizable normal models\footnote{
We thank Daniel Simpson for useful discussion and the Academy of Finland
(grants 298742, 313122) for partial support of this work.
}\vspace{.1in}}
\author{Paul-Christian B\"{u}rkner\footnote{Department of Psychology, University of M\"{u}nster, Germany.}
  \and Jonah Gabry\footnote{Institute for Social and Economic Research and Policy, Columbia University.}
  \and Aki Vehtari\footnote{Department of Computer Science, Aalto University, Finland.
}\vspace{.1in}}
\date{\today \vspace{-.1in}}
\begin{document}\sloppy
\maketitle
\thispagestyle{empty}

\begin{abstract}
Cross-validation can be used to measure a model's predictive accuracy for the
purpose  of model comparison, averaging, or selection. Standard leave-one-out
cross-validation  (LOO-CV) requires the likelihood to be factorizable, but many
important models in temporal  and spatial statistics do not have this property.
We derive how to compute and validate  both exact and approximate LOO-CV for
Bayesian non-factorizable models with a multivariate  normal likelihood.

\textbf{Keywords:} cross-validation, Pareto-smoothed importance-sampling,
  non-factorizable models.
\end{abstract}


\section{Introduction}

%After fitting a statistical model we often want to assess its predictive
%accuracy 
In the absence of new data, cross-validation is a general approach for evaluating
a statistical model's predictive accuracy for the purpose of model comparison, 
averaging, or selection \citep{geisser1979, vehtari2002, ando2010, vehtari2012}.
One widely used variant of
cross-validation is \emph{leave-one-out cross-validation} (LOO-CV), where
observations are left out one at a time and then predicted based on the model
fit to the remaining data. Predictive accuracy is evaluated by first computing
the expected log predictive density of the left-out observation and then taking
the sum of these values over all observations to obtain the expected log
predictive density (ELPD) as a single measure of predictive accuracy.
Unfortunately, exact LOO-CV is costly, as it requires fitting the model as many
times are there are observations in the data. Depending on the size of the data,
complexity of the model, and estimation method, this can be practically
infeasible as it simply requires too much computation time
\citep{vehtari2017loo}. For this reason, approximate versions of LOO-CV have
been developed, most notably using Pareto-smoothed importance-sampling (PSIS,
\cite{vehtari2017loo, vehtari2017psis}), which is applicable to Bayesian models.

A standard assumption of any such LOO-CV approach is that the joint likelihood
of the model over all observations has to be factorizable. That is, the
observations have to be pairwise conditionally independent given the model
parameters. However, many important models do not have this property.
Particularly in the fields of temporal and spatial statistics it is common to
fit models with multivariate normal likelihoods that have structured covariance
matrices such that the likelihood does not factorize.

In this short paper we show how equations derived in \cite{sundararajan2001} can
be repurposed and combined with PSIS to allow for performing efficient
approximate LOO-CV for \emph{any} multivariate normal Bayesian model with an
invertible covariance matrix, regardless of whether or not the likelihood
factorizes. We also provide equations for computing exact LOO-CV for these
models, which can be used to validate the approximation. Throughout, a Bayesian
model specification and estimation via Markov chain Monte Carlo (MCMC) is
assumed. In online supplementary material we provide R code demonstrating how to
carry out the approximation described in the paper as well as comparisons
between the approximate results and results from exact LOO-CV.


\section{Pointwise log-likelihood for non-factorizable normal models}
\label{sec-pointwise}

When computing \emph{exact} LOO-CV for a Bayesian model we need to compute the
log leave-one-out predictive densities $\log{p(y_i | y_{-i})}$ for every
response value $y_i, \: i = 1, \ldots, N$, where $y_{-i}$ denotes all response
values except observation $i$. This requires fitting the model $N$ times. For
\emph{approximate} LOO-CV using only a single model fit, we instead calculate
the pointwise log-likelihood (the pointwise log-predictive density evaluated at
each data point), without leaving out any observations, and then apply an
importance sampling correction \citep{vehtari2017loo}.

The pointwise log-likelihood is straightforward to compute for
\emph{factorizable} models in which response values are conditionally
independent given the model parameters $\theta$ and the likelihood can be
written in the familiar form
%
\begin{equation}
p(y \,|\, \theta) = \prod_{i=1}^N p(y_i \,|\, \theta).
\end{equation}
%
When $p(y)$ can be factorized in this way, the conditional pointwise
log-likelihood can be obtained easily by computing $\log p(y_i \,|\, \theta)$
for each $i$.

The situation is more complicated for \emph{non-factorizable} models in which
response values are not conditionally independent. When there is residual
dependence even after accounting for the model parameters, the conditional
pointwise log-likelihood has the general form $\log p(y_i \,|\, y_{-i},
\theta)$. Most often this is due to the fact that observations depend on other
observations from different time periods or different spatial units. Computing
this pointwise log-likelihood for non-factorizable models is often impossible,
but there is a large class of multivariate normal models for which an analytical
solution is available.

\cite{sundararajan2001} provide equations for the predictive mean and standard
deviation for a zero-mean Gaussian process model with prior covariance $K$ and
residual standard deviation $\sigma$,
%
\begin{equation}
y \sim {\mathrm N}(0, \, K+\sigma^2 I),
\end{equation}
%
where $I$ is the identity matrix of appropriate dimension and $C = K+\sigma^2 I$
is the covariance matrix of the model. These equations were not traditionally
intended to be used for LOO-CV of Gaussian process models because, in most
cases, Gaussian processes are combined with a factorizable likelihood so that
simpler equations for univariate distributions can be applied. But the
derivations of Sundararajan and Keerthi's equations make no use of the special
form of $C$ for Gaussian process models and thus immediately generalize to the
case of an arbitrary invertible covariance matrix $C$.

For such models the LOO predictive mean and standard deviation can be computed
using the equations from \cite{sundararajan2001} as follows:
%
\begin{align}
\label{ypredpars}
  \mu_{\tilde{y},-i} &= y_i-\bar{c}_{ii}^{-1} g_i \nonumber \\
  \sigma_{\tilde{y},-i} &= \sqrt{\bar{c}_{ii}^{-1}},
\end{align}
%
where $g_i = \left[C^{-1} y\right]_i$ and
$\bar{c}_{ii} = \left[C^{-1}\right]_{ii}$.
The log predictive density of the $i$th observation is
%
\begin{equation}
  \log p(y_i \,|\, y_{-i},\theta)
  = - \frac{1}{2}\log(2\pi)
  - \frac{1}{2}\log \sigma^2_{-i}
  - \frac{1}{2}\frac{(y_i-\mu_{-i})^2}{\sigma^2_{-i}},
\end{equation}
%
and expressing this same equation in terms of $g_i$ and $\bar{c}_{ii}$, we
obtain\footnote{ Note that \cite{vehtari2016} has a typo in the corresponding
Equation 34.}:
%
\begin{equation}
\label{logpointwise}
  \log p(y_i \,|\, y_{-i},\theta)
  = - \frac{1}{2}\log(2\pi)
  + \frac{1}{2}\log \bar{c}_{ii}
  - \frac{1}{2}\frac{g_i^2}{\bar{c}_{ii}}.
\end{equation}
%
We can now evaluate equation \eqref{logpointwise} to compute the pointwise
log-likelihood values required for the PSIS-LOO-CV approximation.

\section{Approximate LOO-CV for non-factorizable normal models}
\label{sec-approx}

The conditional pointwise log-likelihood is the only required input to the
PSIS-LOO-CV algorithm from \cite{vehtari2017loo} and thus Sundararajan and
Keerthi's repurposed equations allow for approximate LOO-CV for \emph{any} model
that can be expressed conditionally in terms of a multivariate normal with
invertible covariance matrix $C$, including those where the likelihood does not
factorize. For a Bayesian model fit using MCMC the procedure is as follows:

\begin{enumerate}
\item Fit the model using MCMC obtaining $S$ samples from the posterior
distribution of the parameters $\theta$.
\item For each of the $S$ draws of $\theta$, compute the pointwise
log-likelihood value for each of the $N$ observations in $y$ using
\eqref{logpointwise}. The results can be stored in an $S \times N$ matrix.
\item Run the PSIS algorithm from \cite{vehtari2017loo} on the $S \times N$
matrix obtained in step 2. For convenience the \texttt{loo} R package
\citep{loo2018} provides this functionality.
\end{enumerate}

In the supplementary materials we demonstrate this method by computing
approximate LOO-CV for a lagged simultaneously autoregressive (SAR) model fit to
the spatially correlated crime data.

\section{Validation using exact LOO-CV}

In order to validate the approximate LOO-CV procedure, and also in order to
allow exact computations to be made for a small number of leave-one-out folds
for which the Pareto-$k$ diagnostic \citep{vehtari2017psis} indicates an
unstable approximation, we need to consider how we might to do \emph{exact}
LOO-CV for a non-factorizable model. Here we will provide the necessary
equations and in the supplementary materials we provide code for comparing the
exact and approximate versions.

In the case of a Gaussian process that has the marginalization property, exact
LOO-CV is relatively straightforward: we can simply drop the one row and column
of the covariance matrix $C$ corresponding to the held out observation when
refitting the model. But this does not hold in general for multivariate normal
models, and to keep the original prior we may need to maintain the full
covariance matrix $C$ even when one of the observations is left out.

The general solution is to model $y_i$ as a missing observation and estimate it
along with all of the other model parameters. For a multivariate normal model
$\log p(y_i\,|\,y_{-i})$ can be computed as follows. First, we model $y_i$ as
missing and denote the corresponding parameter $y_i^{\mathrm{mis}}$. Then, we
define
%
\begin{equation}
y_{\mathrm{mis}(i)} = (y_1, \ldots, y_{i-1}, y_i^{\mathrm{mis}}, y_{i+1}, \ldots, y_N).
\end{equation}
%
to be the same as the full set of observations $y$ but replacing $y_i$ with
the \emph{parameter} $y_i^{\mathrm{mis}}$.

Second, we compute the LOO predictive means and standard deviations using the
equations from section \ref{sec-pointwise}, but replacing $y$ with
$y_{\mathrm{mis}(i)}$ in the computation of $\mu_{\tilde{y},-i}$:
%
\begin{equation}
\mu_{\tilde{y},-i} = y_{{\mathrm{mis}}(i)}-\bar{c}_{ii}^{-1}g_i,
\end{equation}
%
where in this case we have
%
\begin{equation}
g_i = \left[ C^{-1} y_{\mathrm{mis}(i)} \right]_i.
\end{equation}
%
The conditional log predictive density is then computed with the above
$\mu_{\tilde{y},-i}$ and the left out observation $y_i$:
%
\begin{equation}
  \log p(y_i\,|\,y_{-i},\theta)
  = - \frac{1}{2}\log(2\pi)
  - \frac{1}{2}\log \sigma^2_{\tilde{y},-i}
  - \frac{1}{2}\frac{(y_i-\mu_{\tilde{y},-i})^2}{\sigma^2_{\tilde{y},-i}}.
\end{equation}
%
Finally, the leave-one-out predictive distribution can be estimated as
%
\begin{equation}
 p(y_i\,|\,y_{-i}) \approx \frac{1}{S} \sum_{s=1}^S p(y_i\,|\,y_{-i}, \theta_{-i}^{(s)}),
\end{equation}
%
where $\theta_{-i}^{(s)}$ are draws from the posterior distribution
$p(\theta\,|\,y_{\mathrm{mis}(i)})$.


\section{Conclusion}

We have provided equations for that enable both approximate and exact LOO-CV for
non-factorizable multivariate normal Bayesian models. Although exact LOO-CV is
usually impractical, our exact LOO-CV procedure can be used to validate the more
efficient PSIS-LOO-CV approximation, as we show in the supplementary materials.

The primary motivation for this paper is to enable approximate LOO-CV for models
that cannot be factorized at all, but our approach also works for \emph{any}
Bayesian model that can be expressed in terms of a multivariate normal
likelihood. Therefore it may also be useful for models that are factorizable but
for which the factorized representation is difficult to compute or not available
to the researcher for some other reason.


\bibliography{psis_non_factorizable_models.bib}
\bibliographystyle{chicago}


%%: appendix
%\clearpage
%\section*{Appendix}

\end{document}
